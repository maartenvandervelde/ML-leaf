\begin{abstract}
\noindent
In the study of plant life, specifically trees, one important characteristic is the recognition of leaves. Due to the wide variety of shapes in which leaves appear, as well as their ease of transportation and photography, they have become one of the chief methods to visually determine the species of a tree or other plant. Modern software applications on smartphones provide a way of performing classification of a leaf, in the field, within seconds of it being photographed.
We have developed a program with the goal of analysing leaf images that were obtained from existing datasets of digitized leaf photographs. Features were extracted from each image using the scale- and rotation-invariant SIFT method, after which the images were described using histograms obtained with a visual Bag-of-Words implementation. These histograms served as input to an artificial neural network which aimed to provide correct identification of the leaves' species. A comparison was made between the neural network and a K-Nearest Neighbour classifier that used the same input. These systems were gauged on accuracy and compared to similar implementations from the literature which differed in both classification method and dataset used.
\end{abstract}